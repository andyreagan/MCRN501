\subsection{Summary}
\begin{itemize}
	\item it is believed tropical convection influences changes in SST (model)
	\item SPCZ GCMs and model do not correctly account for the orientation of SPCZ (model and data)
	\item ITCZ is modeled unrealistically in several prominent models (data)
	\item South Pacific island rainfall explained by ITCZ (model and data)
	\item SACZ explains flood/dry conditions  in south east Brazil (data) 
	\item ENSO may display variability explained by natural variability or by forcing. Models mimc observed data without external forcing, but better understanding of the underlying mechanisms is required to justify this assumption. 
	\item ENSO will remain a dominant mode of inter annual variability with global influences. Namely, ENSO induces rainfall intensity variability because of increased moisture 
\end{itemize}

\subsection{Wants}
\begin{itemize}
	\item there is a demand for improved modeling of SST warming patterns to improve rainfall estimates in climatological convergence zones
	\item improve annual estimates of SST warming patterns to the confidence of seasonal estimates
		\subitem{-} (COMPUTATIONAL SCIENTIST) improvements in prediction can be achieved with more computationally expensive model runs
	\item correct modeled orientation of SPCZ
		\subitem{-} (COMPUTATIONAL SCIENTIST) seems fair enough. I do not imagine this being computationally difficult
	\item more knowledge of zonally-oriented SPCZ events
	\item better knowledge of IOB formation from ENSO
	\item better understanding of annual SST change (data taken from coral isotopes)
	\item model for tropical Atlantic Climate variability in boreal summer
		\subitem{-} possibly incorporate feedback loops between surface wind, evaporation, and SST
	\item better predictions of AMM activity
	\item better MJO modeling 
		\subitem{-} sensitive to SST warming pattern, which itself has high uncertainty 
	\item demand for better understanding of how a warmer climate influences ENSO

\end{itemize}
